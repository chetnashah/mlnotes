\documentclass[10pt,a4paper]{article}
\usepackage[latin1]{inputenc}
\usepackage{amsmath}
\usepackage{amsfonts}
\usepackage{amsthm}
\usepackage{amssymb}
\usepackage{graphicx}
\begin{document}
	\theoremstyle{definition}
	\newtheorem{defn}{Definition}[section]
	
	\section{Calculus}
	\begin{defn}
		\textbf{Continuity at a point} $x = a$:\\
			\fbox{
					A function $f$ is continuous at $x = a$ if 
		$\lim\limits_{x \to a} f(x) = f(a)$	
		}

	\end{defn}
	There are three subtle points that should not be missed:
	\begin{itemize}
		\item Two sided limit $\lim\limits_{x \to a} f(x)$ exists and is finite
		\item The function is defined at $x = a$, i.e. $f(a)$ exists
		\item The above two quantities are equal i.e 
		$\lim\limits_{x \to a} f(x) = f(a)$	
	\end{itemize}.

	Some of the cases the continuity fails is, e.g. two sided limit exists but $f(a)$ has a different value, or two sided limit exists and $f(a)$ is not defined.\\ \\ In essence, for $f$ to be continuous at a point $x = a$, two sided limit should exist and be same as $f(a)$ at $x = a$.
	
	\begin{defn}
		\textbf{Continuity on an interval}\\
		$f$ is continuous on the interval $(a,b)$ if it is continuous
		at every point in the interval.
		
		
	\end{defn}

	\begin{defn}
	\textbf{Continuity and Intermediate Value Theorem(IVT)}\\
	If $f$ is continuous on $[a,b]$, and $f(a) < 0$ and $f(b) > 0$,\\ then there is at least one number $c$ in the interval $(a,b)$ such that
	$f(c) = 0$.\\ \\ The same is true if instead $f(a) > 0$ and $f(b) < 0$.	
	
\end{defn}

	\begin{defn}
	\textbf{Continuity and Max-Min Theorem}\\
	If $f$ is continuous on $[a,b]$,\\
	then f has atleast one maximum and one minimum on $[a,b]$.	
	
\end{defn}
	
\end{document}