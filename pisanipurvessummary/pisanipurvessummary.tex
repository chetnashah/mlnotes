\documentclass[10pt,a4paper]{article}
\usepackage[latin1]{inputenc}
\usepackage{amsmath}
\usepackage{amsfonts}
\usepackage{amssymb}
\usepackage{graphicx}
\begin{document}
	\section{Chapter 1}
	\section{Chapter 2}
	\section{Chapter 3. The Histogram}
	
	\begin{enumerate}
		\item In a histogram, the areas of blocks represent percentages, and the total area under histogram should be 100%
		\item To figure out height of of block over class interval, divide percentage by length of interval.
		\item Standard convention for histograms is X-Axis : Property in some units, Y-Axis: Percent per Unit.
		\item The height of histogram represents crowding in that particular interval.
		\item A variable is a characteristic of the subjects in study. It can be either qualitative or quantitative(discrete or continuous).
		\item A co-founding factor is sometimes controlled for by cross-tabulation.	
	\end{enumerate}
	
	\section{Chapter 4. Average and Standard Deviation}
	
	\begin{enumerate}
		\item The average of a list of numbers is equal to sum divided of how many there are.
		\item The median is a positional entity, with half of observations falling greater and rest half of observation falling lower than itself.
		\item RMS of a list is root of the mean of the square of items in the list. i.e. root(mean(sqr(items)))
		\item SD of a list is root of the mean of the square of deviation of items from original avg. i.e. root(mean(sqr(deviations)))
		\item Roughly 68\% of entries on a list are within one SD of average, Roughly 95\% are within 2 SDs of avg. (This assumption only holds for data that can be approximated by a normal curve)
		
	\end{enumerate}
	
\end{document}